\documentclass{article}
\usepackage[polish]{babel}
\usepackage[T1]{fontenc}
\usepackage{geometry}
\usepackage{chngpage}
\usepackage{graphicx}
\usepackage{subcaption}
\usepackage{algorithm2e}
\usepackage{amsfonts}
\graphicspath{ {./plots/} }
\geometry{margin=2cm}
\usepackage[utf8]{inputenc}
\usepackage{indentfirst}
\usepackage{longtable}
\author{Benjamin Jurczok - 244760}
\title{\vspace{-2.0cm}Zadanie 1 z listy 6}
\frenchspacing
\setlength{\parindent}{2em}
\usepackage{bm}
\newcommand{\mA}{\bm{A}}
\newcommand{\mB}{\bm{B}}
\newcommand{\mC}{\bm{C}}
\newcommand{\mL}{\bm{L}}
\newcommand{\mU}{\bm{U}}
\newcommand{\mZ}{\bm{0}}
\newcommand{\vb}{\bm{b}}
\newcommand{\vx}{\bm{x}}
\newcommand{\R}{\mathbb{R}}

\begin{document}
\maketitle
\section*{Treść zadania:}
\noindent Rozważmy gramatykę:
$$S \rightarrow E$$
$$E \rightarrow E \hspace{0.2cm} or \hspace{0.2cm} T|T$$
$$T \rightarrow T \hspace{0.2cm} and \hspace{0.2cm} F|F$$
$$F \rightarrow \hspace{0.2cm} not \hspace{0.2cm} F|(E)|true|false$$
Pokazać że gramatyka generuje formuły boolowskie ze stałymi true and false, oraz odpowiedzieć na pytanie czy gramatyka jest jednoznaczna.
\section*{Rozwiązanie:}
\noindent \textbf{Generowanie formuł boolowskich:}\\\\
\noindent Twierdzenie udowodnię za pomocą dowodu indukcyjnego po słowie długości \textit{n} wygenerowanego przez gramatykę.
\begin{enumerate}
	\item Dla \textit{n}=1 otrzymujemy \textit{true} lub \textit{false}, a więc jest f. boolowską
	\item Dla słów długości < \textit{n} załóżmy że także otrzymamy formy boolowskie, wtedy wystaczy pokazać że dla \textit{n} także je otrzymamy.
	\begin{itemize}
		\item Gdy słowo kończy się ' ) ', więc została użyta produkcja $F \rightarrow (E)$. Słowo z pochodzące z kolejnej produkcji (od E) będzie więc długośc n-2 i więc z założenia indukcyjnego będzie formułą boolowską
		\item Słowo kończy się \textit{true} albo \textit{false} i użyta była produkcja $E \rightarrow E \hspace{0.2cm} or \hspace{0.2cm} T|T$ albo $T \rightarrow T \hspace{0.2cm} and \hspace{0.2cm} F|F$. Slowo generowane przez E lub T jest długości n-2, więc założenia indukcyjnego słowa wygenerowane przez E i T są formułą boolowską, zatem słowo o długości \textit{n} też będzie.
	\end{itemize}
\end{enumerate}
\vspace{1cm}
\noindent \textbf{Jednoznaczność gramatyki:}\\\\
\noindent Jednoznaczność udowodnię konstruując tablicę parsera LL(1).\\\\
\noindent Na początku eliminuję lewostronną rekursję z gramatyki:\\\\
$S \rightarrow E$\\
$E \rightarrow TE'$\\
$E' \rightarrow or \hspace{.2cm} TE'|\epsilon$\\
$T \rightarrow FT'$\\
$E' \rightarrow and \hspace{.2cm} FT'|\epsilon$\\
$F \rightarrow not \hspace{.2cm} F|(E)|true|false$\\
\newpage
\noindent Wyznaczam FIRST and FOLLOW dla gramatyki:\\\\
$FIRST(E) = \{not,(,true,false\}$\\
$FIRST(T) = \{not,(,true,false\}$\\
$FIRST(F) = \{not,(,true,false\}$\\
$FIRST(E') = \{or,\epsilon\}$\\
$FIRST(T') = \{and,\epsilon\}$\\
$FOLLOW(E) = \{\$,)\}$\\
$FOLLOW(E') = \{\$,)\}$\\
$FOLLOW(T) = \{\$,),or\}$\\
$FOLLOW(T') = \{\$,),or\}$\\
$FOLLOW(F) = \{\$,),and,or\}$\\
\noindent Konstruuję tablicę parsera:\\\\
\begin{center}
	\begin{tabular}{|p{1cm}|p{1cm}|p{1cm}|p{1cm}|p{1.5cm}|p{1cm}|p{1cm}|p{1cm}|p{1cm}|}
		\hline
		 & \textbf{true} & \textbf{false} & \textbf{not} & \textbf{and} & \textbf{or} & \textbf{(} & \textbf{)} & \textbf{\$} \\
		\hline
		 E & TE' & TE' & TE' &  &  & TE' &  & \\
		  \hline
		 E' &  &  &  &  & or TE' &  & $\epsilon$ & $\epsilon$ \\
		  \hline
		 T & FT' & FT' & FT' &  &  & FT' &  & \\
		  \hline
		 T' &  &  &  & and FT' & $\epsilon$ &  & $\epsilon$ & $\epsilon$ \\
		  \hline
		 F & true & false & not F &  &  & (E) &  &  \\
		\hline
	\end{tabular}
\end{center}
W tablicy nie ma konfliktów - gramatyka jest jednoznaczna.
\end{document}